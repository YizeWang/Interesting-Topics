\documentclass[12pt]{article}

\usepackage{sectsty}
\usepackage{graphicx}
\usepackage{amsmath}
\usepackage{amssymb}

% Margins
\topmargin=-0.45in
\evensidemargin=0in
\oddsidemargin=0in
\textwidth=6.5in
\textheight=9.0in
\headsep=0.25in

\title{Gender Ratio Problem}
\author{Yize Wang}
\date{\today}

\begin{document}
\maketitle
\pagebreak

\section{Problem Statement}
Theoretically, a mother gives birth to a boy or girl with the same probability. The gender ratio of boys and girls is thus expected to be $0.5:0.5$ in a large population. What if every mother will not stop having a new child until she has a boy? Will the ratio be changed?

\section{Solution}
Let $N$ denote the number of mothers and assume they have new children simultaneously. There will be $N/2$ boys and girls in the first birth wave. For the $N/2$ mothers who do not have boys will have new $N/2/2$ boys and girls, and so on. \\
The total number of girls $N_G$ can be obtained by the geometric series sum formula:
\begin{align*}
	N_G = \sum_{i}\frac{1}{2^i}N = \left( 1-\left( \frac{1}{2} \right)^i \right) N.
\end{align*}
Since each mother eventually has one and only one boy, the total number of boys is the same as mothers, i.e., $N_B = N$.\\
The gender ratio $R$ is thus
\begin{align*}
	R =\frac{ N_G}{N_B} = \left( 1-\left( \frac{1}{2} \right)^i \right).
\end{align*}
We list the ratio values below for small $N$.
\begin{center}
	\begin{tabular}{|c| c c c c c|}
		\hline
		$N$ & 1           & 2           & 5           & 10          & 100         \\
		\hline
		$R$ & $0.33:0.67$ & $0.43:0.57$ & $0.49:0.51$ & $0.50:0.50$ & $0.50:0.50$ \\
		\hline
	\end{tabular}
\end{center}
Counter-intuitively, as $N$ grows, the ratio quickly approaches $0.5:0.5$. Therefore, in a large population, the evil reproductive policy above will not change gender ratio. \\
However, since boys are always the last child in a family, the average age of boys will be less than that of girls, which will affect gender ratio in a long term (takes generations).


\end{document}