\documentclass[12pt]{article}

\usepackage{sectsty}
\usepackage{graphicx}
\usepackage{amsmath}
\usepackage{amssymb}

% Margins
\topmargin=-0.45in
\evensidemargin=0in
\oddsidemargin=0in
\textwidth=6.5in
\textheight=9.0in
\headsep=0.25in

\title{Derangement Problem}
\author{Yize Wang}
\date{\today}

\begin{document}
\maketitle
\pagebreak

\section{Problem Statement}
Assume there are $N$ birds and $N$ nests both indexed from $1$ to $N$. How many unique permutations are there if each bird wants to live in a nest with a different index from its own. In other words, the $k^{th}$ bird does not want to live in the $k^{th}$ nest.

\section{Solutions}
An algebraic solution and a brute-force backtrace solution are offered in this section.
\subsection{Algebraic Solution}
Let $D_N$ denote the number of permutations when there are $N$ birds and $N$ nests. \\ \\
$N = 1$, the bird has to stay in its own nest, and no permutation qualifies, i.e., $D_1 = 0$; \\
$N = 2$, the only permutation is that the two birds uses the other nest, i.e., $D_2 = 1$; \\
$N > 2$, we consider two steps: Let Bird $1$ decide first and it has $N-1$ options. Without loss of generality, assume Bird $1$ picks Nest $2$; Next step, Bird $2$ can either pick Nest $1$ (case $A$) or Nest $\{3,\dots,N\}$ (case $B$). In case $A$, Bird $\{3,\dots,N\}$ should pick their nests from Nest $\{3,\dots,N\}$, which results in $D_{N-2}$ permutations (sub-problem of $N-2$ birds and nests). In case $B$, Bird $2$ cannot pick Nest $1$ which would belong to case $A$. Then Bird $2$ has to pick a nest from Nest $\{3,\dots,N\}$, and Bird $\{3,\dots,N\}$ has to choose from $\{1,3,\dots,N\}$. If we rename Nest $1$ to Nest $2$, nothing will be changed for this sub-problem, which is equivalent to solving $D_{N-1}$. \\ \\
The permutations in step $1$ and step $2$ should be multiplied, and thus we obtain the recursive formula:
\begin{align*}
	D_1   & = 0                                                                                      \\
	D_2   & = 1                                                                                      \\
	D_{N} & = \left( N - 1 \right) \left( D_{N-1} + D_{N-2} \right), \forall N > 2, N \in \mathbb{Z}
\end{align*}
We list below some results for small $N$s.
\begin{center}
	\begin{tabular}{|c| c c c c c c c c c c c c|}
		\hline
		$N$   & 1 & 2 & 3 & 4 & 5  & 6   & 7    & 8     & 9      & 10      & 11       & 12        \\
		\hline
		$D_N$ & 0 & 1 & 2 & 9 & 44 & 265 & 1854 & 14833 & 133496 & 1334961 & 14684570 & 176214841 \\
		\hline
	\end{tabular}
\end{center}
Math masters also derived general formula for $D_N$. Interested readers are referred to Wikipedia by keyword \textit{Derangement} for more details. Programmers may also want to checkout \textit{Leetcode 634 Find the Derangement of an Array} for more insights.

\subsection{Backtrace Solution}
We can also utilize modern computer's super power to iterate all possible permutations. The implementation can be found in \textit{main.cpp}. The algorithm's space and time complexity are $\mathcal{O}\left( n^2 \right)$ and $\mathcal{O}\left( n \times n!\right)$ respectively. \\

\section{Acknowledgment}
Thank Chengzhi Qi brought up this problem and offered insights.
\end{document}